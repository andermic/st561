\documentclass{article}
\usepackage[pdftex]{graphicx}
\usepackage{amsmath}
\usepackage{verbatim}
\usepackage{enumerate}
\author{Michael Anderson}
\title{Homework 2}
\begin{document}
\setlength{\parskip}{1em}
\maketitle
\center{ST561}
\center{Prof. Jiang}\\
\flushleft
\newpage

\section{}
\begin{enumerate}[(a)]
\item
The probability of drawing the first ace is the number of aces in the deck (4)
divided by the number of cards left in the deck (52). For the second ace, we
have the number of aces left in the deck divided by the number of cards left in
the deck (51) ... and so on down to the last ace:

\[
\frac{4 \times 3 \times 2 \times 1}{52 \times 51 \times 50 \times 49} = 
\frac{1}{270725}
\]

\item
Here the size of the deck decreases as cards are drawn, as in (a), but only one
card in valid for each draw giving:

\[
\frac{1^4}{52 \times 51 \times 50 \times 49} = 
\frac{1}{6497400}
\]

\item
For (a), both the number of aces left in the deck and the total number of cards 
left in the deck does not change after each draw, giving:

\[
\frac{4 \times 3 \times 2 \times 1}{52^4} = 
\frac{3}{913952}
\]

Similarly, (b) with replacement becomes:

\[
\frac{1^4}{52^4} = 
\frac{1}{7311616}
\]

\end{enumerate}

\section{}
There are 4! ways to permute the 4 different subjects on the shelf.
Within each subject there are 6!, 4!, 4!, 5! ways to permute the spanish,
history, geology, and english books, respectively.

\[
4! \times 6! \times 4! \times 4! \times 5! = 1194393600 
\]

\section{}
\begin{enumerate}[(a)]
\item
The men could be in 6 groups of contigous seats: 1-5, 2-6, 3-7, 4-8, 5-9, 6-10.
Within those seats, they could be permuted in 5! ways. $6 \times 5! = 720$.

\item
The men could be in the odd-numbered seats, and the women in the even-numbered,
or vice versa. The men could be permuted in their seats in 5!, and so could the
women. $2 \times 5! \times 5! = 28800$.

\item
There are 5 couples, which can be permuted in 5! ways. Within each couple, the
man could sit to the woman's right, or vice versa. $5! \times 2^5 = 3840$. 

\item
The sample space is the permutation of all 10 attendees, of which we are
interested in 3840:

\[
\frac{3840}{10!} = \frac{1}{945}
\]

\end{enumerate}

\section{}
\begin{enumerate}[(a)]
\item
There are $n$ possible places to put each of the $r$ balls, so the sample space
is $n^r$. There are $r!$ ways to permute the different balls in the first $r$
urns, so the probability is $r!/n^r$.

\item
Since the balls are different, there are $P^n_r$ ways to place $r$ of them
into their own of $n$ urns. The probability is then $P^n_r/n^r$

\item
There are ${{r}\choose{m}}$ ways to place the balls in $U_1$, since it is not
specified that the order that they are placed into $U_1$ is important. Each of
the remaining $r-m$ balls can be placed into any of the $n-1$ urns, in
$(n-1)^{r-m}$ ways. So the probability is:

\[
\frac{{{r}\choose{m}}(n-1)^{r-m}}{n^r}
\]

\end{enumerate}

\section{}
\begin{enumerate}[(a)]
\item
$P(A)$ is correct, and $P(B)$ is incorrect.
\item
\[
P(B) = \frac{{{13}\choose{2}} {{4}\choose{2}} {{4}\choose{2}} (44)}
{{{52}\choose{5}}} = 
\frac{\frac{(13)(12)}{2} {{4}\choose{2}} {{4}\choose{2}} (44)}
{{{52}\choose{5}}}
\]
\item
For full houses, there are 13 possibilities for the 3-card rank, and 12
possibilities for the 2-card rank. There are ${{4}\choose{3}}$ ways to select
suits for the 3-card rank, and ${{4}\choose{2}}$ ways to select suits for the
two card rank. Finally the sample space is ${{52}\choose{5}}$.

For two pairs, there are ${{13}\choose{2}}$ ways to select 2 ranks from the 13
available. This is because the order that the ranks for the pairs are
selected in does not matter, unlike full houses
where for example $\{JJJ99\} \ne \{999JJ\}$. There are ${{4}\choose{2}}$ ways
to select suits for one pair, as well as ${{4}\choose{2}}$ ways to select suits
for the other pair. For the fifth card, there are $52-(4 \times 2) = 44$ cards
left in the deck that do not have the same rank as either of the pairs. Finally,
the sample space is again ${{52}\choose{5}}$.

\end{enumerate}
\end{document}
