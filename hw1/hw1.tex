\documentclass{article}
\usepackage[pdftex]{graphicx}
\usepackage{amsmath}
\usepackage{verbatim}
\usepackage{enumerate}
\author{Michael Anderson}
\title{Homework 1}
\begin{document}
\setlength{\parskip}{1em}
\maketitle
\center{ST561}
\center{Prof. Jiang}\\
\flushleft
\newpage

\section{Problem 5}
Proof by induction:

\begin{itemize}
\item{\textbf{Base case}.}
It is given in the hint to this problem that we should start at the $n = 3$
case, which we already know works:
\[
P(A_1 \cup A_2 \cup A_3) = \sum_{i=1}^3 P(A_i) \hspace{6px} - \sum_{i=1, j>i}^3 
P(A_i \cap A_j) \hspace{6px} + \sum_{i=1, k>j>i}^3 P(A_i \cap A_j \cap A_k) =
\]

\[
P(A_1) + P(A_2) + P(A_3) - P(A_1 \cap A_2) - P(A_1 \cap A_3) - P(A_2 \cap A_3)
+ P(A_1 \cap A_2 \cap A_3)
\]

\vspace{1em}

\item{\textbf{Inductive case}.}
Want to assume that:
\[
P(\cup_{i=1}^n A_i) = \sum_{i=1}^n P(A_i) - \sum_{j>i, i=1}^n
P(A_i \cap A_j) + \sum_{k>j>i, i=1}^n P(A_i \cap A_j \cap A_k) \cdots
+(-1)^{n-1} P(\cap_{i=1}^n A_i)
\]

To show that:
\[
P(\cup_{i=1}^{n+1} A_i) = \sum_{i=1}^{n+1} P(A_i) - \sum_{j>i, i=1}^{n+1}
P(A_i \cap A_j) + \sum_{k>j>i, i=1}^{n+1} P(A_i \cap A_j \cap A_k) \cdots
+(-1)^{n} P(\cap_{i=1}^{n+1} A_i)
\]

Here we must consider how to calculate the probability of the union of $A_{n+1}$
with the other $n$ sets. If we
simply add in the probability, then we will be double-counting all of the
elements in the intersection of $A_{n+1}$ with the other $n$ sets. If we then
subtract off that
difference, we will subtract off all of the places in which $A_{n+1}$
intersects with two other sets twice, which is one time too many. So we must add
those intersections
back in, but then we are double-counting all of the places in which $A_{n+1}$
intersects with three sets. So we must subtract that off... and this pattern
continues up to $n$. This insight, along with the inductive assumption, gives:

\[
P(\cup_{i=1}^n A_i) \cup P(A_{n+1})= \sum_{i=1}^n P(A_i) - \sum_{j>i, i=1}^n
P(A_i \cap A_j) + \sum_{k>j>i, i=1}^n P(A_i \cap A_j \cap A_k) \cdots
+(-1)^{n-1} P(\cap_{i=1}^n A_i)
\]

\[
+ P(A_{n+1}) - \sum_{i=1}^n P(A_{n+1} \cap A_i) + \sum_{j>i,i=1}^n P(A_{n+1}
\cap A_i \cap A_j) \cdots +(-1)^{n} P(\cap_{i=1}^{n+1} A_i)
\]

The two terms on the left side of the equation are easy to combine, and the
terms on the right side can be combined in pairs by like number of intersections
to give:

\[
P(\cup_{i=1}^{n+1} A_i) = \sum_{i=1}^{n+1} P(A_i) - \sum_{j>i, i=1}^{n+1}
P(A_i \cap A_j) + \sum_{k>j>i, i=1}^{n+1} P(A_i \cap A_j \cap A_k) \cdots
(-1)^{n} P(\cap_{i=1}^{n+1} A_i)
\]

\end{itemize}

\section{Problem 6}
Assume that each of the possible 4 combinations of two children ($BB$, $BG$,
$GB$, $GG$) are equally likely.

In the case of the man, we are given the information that at least one of
his children is a boy, or alternatively that he does not have two girls. We have
then 3 equally likely possibilities remaining ($BG$, $GB$, $BB$), of which only
one is two boys. So the probability that he has two boys is $\frac{1}{3}$.

In the case of the woman, we are given that her second child is a boy. We have
then 2 equally likely possibilities remaining ($BB$, $BG$), of which only one
is two boys. So the probability that she has two boys is $\frac{1}{2}$.

So no, the probabilities are not equal.

\section{Problem 7}
\begin{enumerate}[(a)]
\item
Let:

\vspace{3px}

\begin{tabular}{l l}
$P(C) = \frac{1}{5}$ & be the probability that a taxpayer cheated \\
$P(C^C) = \frac{4}{5}$ & be the probability that a taxpayer did not cheat \\
$P(Y|R) = 1$ & be the probability that a taxpayer said ``Yes" given that
they rolled red \\
$P(Y|C) = 1$ & be the probability that a taxpayer said ``Yes" given
that they cheated \\
$P(Y|C^C) = \frac{1}{5}$ & be the probability that a taxpayer said ``Yes" given
that they did not cheat \\
\end{tabular}

\vspace{1em}

Want to find $P(C|Y)$, the probability that a taxpayer cheated given that they
said ``Yes". By Bayes' Theorem:

\[
P(C|Y) = \frac{P(Y|C)P(C)}{P(Y)} = 
 \frac{P(Y|C)P(C)}{P(Y|C)P(C) + P(Y|C^C)P(C^C)} =
 \frac{1(1/5)}{1(1/5) + (1/3)(4/5)} = \frac{3}{7} 
\]

\item
We get no information from the approximately $120 \times \frac{1}{3} = 40$
taxpayers who said said ``Yes" automatically because they rolled red. Out of the
approximately $120 - 40 = 80$ taxpayers who rolled green, $54 - 40 = 14$ said
``Yes" and therefore cheated. So the estimate is then $14/80$.

\end{enumerate}

\end{document}
